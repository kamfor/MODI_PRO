% TeX encoding = utf8
% TeX spellcheck = pl_PL 
\documentclass[a4paper, 11pt]{article}
\usepackage[utf8]{inputenc}
\usepackage[polish]{babel}
\usepackage{polski}
\usepackage{float}
\usepackage{graphicx}
\usepackage{listings}
\usepackage{amsfonts}
\usepackage{geometry}
\usepackage{multicol}
\usepackage{latexsym}
\usepackage{enumerate}
\usepackage{hyperref}
\usepackage{wrapfig}
\usepackage{color} %red, green, blue, yellow, cyan, magenta, black, white
\definecolor{mygreen}{RGB}{28,172,0} % color values Red, Green, Blue
\definecolor{mylilas}{RGB}{170,55,241}

\author{Kamil Foryszewski}
\title{Modelowanie i identyfkacja - projekt II, zadanie 9}
\frenchspacing

\newgeometry{tmargin=2cm, bmargin=2cm, lmargin=2cm, rmargin=2cm}
\pagestyle{empty}


\begin{document}

\lstset{language=Matlab,%
    basicstyle=\color{red},
    breaklines=true,%
    morekeywords={matlab2tikz},
    keywordstyle=\color{blue},%
    morekeywords=[2]{1}, keywordstyle=[2]{\color{black}},
    identifierstyle=\color{black},%
    stringstyle=\color{mylilas},%
    commentstyle=\color{mygreen},%
    showstringspaces=false,
    numbers=right,%
    numberstyle={ \color{black}},% size of the numbers
    numbersep=5pt, % this defines how far the numbers are from the text
    emph=[1]{for,endfor,endwhile,endfunction,endif,break},emphstyle=[1]\color{blue}, %some words to emphasise
    emph=[2]{,.}, emphstyle=[2]\color{yellow},%
}

\maketitle
\tableofcontents

\section{Identyfikacja modelu statycznego}
\subsection{Wykres danych ststycznych}
\subsection{Podział daych statycznych na zbiory}
\subsubsection{Reprezentacja graficzna zbioru uczącego}
\subsubsection{reprezentacj graficzna zbioru weryfikującego}
\subsection{Model liniowy}
\subsubsection{Wykres charakterystyki modelu liniowego}
\subsubsection{Błędy zbiorów}
\subsubsection{Wykres charakterystyki na tle zbiorów danych}
\subsection{Statyczne modele nielioniwe}
\subsubsection{Model stopnia drugiego}
\subsubsection{Model stopnia trzeciego}
\subsubsection{Model stopnia czwartego}
\subsubsection{Model stopnia piątego}
\subsubsection{Porównanie modeli nieliniowych}
\subsection{Wybór najlepszego modelu}


\section{Identyfikacja modelu dynamicznego}
\subsection{Reprezentacja graficzna danych dynamicznych}
\subsubsection{Zbiór uczący}
\subsubsection{Zbiór weryfikujacy}
\subsection{Dynamiczne modele liniowe}
\subsubsection{Rzędu pierwszego}
\subsubsection{Rzędu drugiego}
\subsubsection{Rzedu trzeciego}
\subsubsection{Porównianie modeli dynamicznych}
\subsection{Wybór najlepszego modelu}
\subsection{Modele dynamiczene wyznaczne metodą najmniejszych kwadratów} 
\subsubsection{Modele o dynamice pierwszego rzędu}
\subsubsection{Modele o dynamice drugirgo rzędu}
\subsubsection{Modele o dynamice trzeciego rzędu}
\subsubsection{Modele mieszane}
\subsubsection{Porównanie modeli} %tabelka
\subsection{Wybór najlepszego modelu nieliniowego}


\section{Wyznaczanie statycznego modelu nieliniowego}
\subsection{Statyczny model nieliniowy}
\subsection{Reprezentacja graficzna statycznego modelu nieliniowego}



\end{document}